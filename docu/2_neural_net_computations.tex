\subsection{Neural Net Computations}

\subsubsection*{Initialize the network}

\begin{itemize}
    \item Set up the network structure by defining:
    \begin{itemize}
        \item total number of layers $L$
        \item nodes per layer
        \item activation functions for hidden and output layers
    \end{itemize}
    \item Initialize the weights $w$ and biases $b$ with random values.
    \item Read the training data and the corresponding labels.
\end{itemize}

\subsubsection*{Forward pass}

\begin{enumerate}
    \item Make $a_t^0$ available for every $t$ in the input layer ($l=0$) for a given
    training input $\vec{x}_i$.
    \item Compute the activation $z_t^l = \sum\limits_{f} w_{tf}^l a_f^{l-1} + b_t^l$ for
    every node in a forward pass for each t in each layer $l = [1,L-1]$ starting from
    $l=1$.
    \item Make the output for each layer available by computing $a_t^l = f_h(z_t^l)$ for
    the hidden layers and $a_t^l = f_o(z_t^l)$ for the output layer.
    \item Calculate the partial loss $L_i$ for the given training pair $(\vec{x}_i,
    \vec{y}_i)$ and the calculated output from the forward pass.
\end{enumerate}

\subsubsection*{Loss function derivatives}

The total loss $L = \frac{1}{N}\sum\limits_i^{N}{L_i}$ is an average of the partial losses
$L_i$ for the given training pairs $(\vec{x}_i, \vec{y}_i)$. To calculate how the loss
function depends on the chosen learning parameters, we use the chain rule:
\begin{subequations}
    \begin{align}
        \pd{L_i}{w_{tf}^l} & =
        \underbrace{\pd{L_i}{z_t^l}}_{\equiv\,\delta_t^l} \pd{z_t^l}{w_{tf}^l}
        = \delta_t^l \pd{z_t^l}{w_{tf}^l}
        \label{eq:loss_gradient_by_weight} \\
        \pd{L_i}{b_t^l} & =
        \underbrace{\pd{L_i}{z_t^l}}_{\equiv\,\delta_t^l} \pd{z_t^l}{b_t^l}
        = \delta_t^l \pd{z_t^l}{b_t^l}
        \label{eq:loss_gradient_by_bias}  
    \end{align}
\end{subequations}
$\delta_t^l = \pd{L_i}{z_t^l}$ is called the "error term" due to the fact that it is
directly linked to the deviation between actual and computed output of the neural net in
the forward pass. \\

Using the definition of $z_t^l = \sum_{f}{w^l_{tf} a^{l-1}_f} + b^l_t$
from the previous section we get
\begin{subequations}
    \begin{align}
    \pd{z_t^l}{w^l_{tf}} & =\pd{}{w^l_{tf}}\left(
        \sum\limits_{f}{w^l_{tf} a^{l-1}_f} + b^l_t
        \right) = a^{l-1}_f \label{eq:activation_gradient_by_weight} \\
    \pd{z_t^l}{b^l_t} & =\pd{}{b^l_t}\left(
        \sum\limits_{f}{w^l_{tf} a^{l-1}_f} + b^l_t
        \right) = 1 \label{eq:activation_gradient_by_bias}
\end{align}
\end{subequations}
where only one term remains of the sum for specific values for $t$ and $f$
in~(\ref{eq:activation_gradient_by_weight}). \\

Combining the previous equations we get as a result
\begin{subequations}
    \begin{align}
        \pd{L_i}{w_{tf}^l} & = \delta_t^l \cdot a^{l-1}_f
        \label{eq:loss_gradient_by_weight_condensed}\\
        \pd{L_i}{b^l_t} & = \delta_t^l
        \label{eq:loss_gradient_by_bias_condensed}
    \end{align}
\end{subequations}

\subsubsection*{Backpropagation - output layer ($l = L-1$):}

With
\begin{equation}
    L_i = L_i(\vec{x}_i, \vec{y}_i) = \sum\limits_t L_i(a_t^l,y_t)
    = \sum\limits_t L_i \left( f_o(z_t^l),y_t \right)
\end{equation}
and using the definition of the error term $\delta_t^l$ from above we get
\begin{equation}
    \delta_t^l \equiv \pd{L_i}{z_t^l} = \sum\limits_t \pd{L_i}{f_o}
    \left( f_o(z_t^l),y_t \right) \cdot \pd{f_o}{z_t^l}
    = \sum\limits_t L_i^\prime \left( f_o(z_t^l),y_t \right)
    \cdot f_o^\prime(z_t^l)
\end{equation}
which, when inserting into equations (\ref{eq:loss_gradient_by_weight_condensed}) and
(\ref{eq:loss_gradient_by_bias_condensed}), yields
\begin{subequations}
    \begin{align}
        \pd{L_i}{w_{tf}^l} & = \delta_t^l \cdot a^{l-1}_f =
        \sum\limits_t L_i^\prime \left( f_o(z_t^l),y_t \right)
        \cdot f_o^\prime(z_t^l) \cdot a^{l-1}_f 
        \label{eq:loss_gradient_by_weight_expanded}\\
        \pd{L_i}{b^l_t} & = \delta_t^l = 
        \sum\limits_t L_i^\prime \left( f_o(z_t^l),y_t \right)
        \cdot f_o^\prime(z_t^l)
        \label{eq:loss_gradient_by_bias_expanded}
    \end{align}
\end{subequations}
These formulas will be directly coded in the backwards pass during the training cycle to
calculate the gradients for learning. Why it makes sense to combine the derivative of the
loss function $L_i^\prime$ and the derivative of the activation function in the output
layer $f_o^\prime(z_t^l)$ can be seen at this URL:\\
\tiny
\url{https://davidbieber.com/snippets/2020-12-12-derivative-of-softmax-and-the-softmax-cross-entropy-loss/}
\normalsize \\
for the combination of cross-entropy loss and the softmax function as an example. The
combination significantly simplifies the resulting computation, when compared to computing
both derivatives separately.

\subsubsection*{Backpropagation - hidden layers ($1 \leq l \leq L-2$):}

For hidden layers we try to link the local contribution to the loss in layer $l$ with the
contribution to the loss in a layer $l+1$. Like in the forward pass we have to refer to
different layers with in the formulas to link them\footnote{In the notation we use here,
again the index $t$ stands for \emph{to}, whereas $f$ stands for \emph{from}. However,
index $t$ here refers to nodes in layer $l+1$, while index $f$ refers to nodes in layer
$l$ for the backward pass. This is conceptually the same use as for the forward pass, but
with reference to other layers.}. \\

Looking at the error term and using the chain rule again we can find a link between
different layers:
\begin{equation}
    \delta_f^l = \pd{L_i}{z_f^l}
    = \sum\limits_t \underbrace{\pd{L_i}{z_t^{l+1}}}_{= \delta_t^{l+1}} \pd{z_t^{l+1}}{z_f^l} 
    = \sum\limits_t \delta_t^{l+1} \pd{z_t^{l+1}}{z_f^l}
    \label{eq:delta_hidden}
\end{equation} 

Using the definition of $z_t^{l+1}$ we get
\begin{equation}
    z_t^{l+1} = \sum\limits_f \left( w_{ft}^{l+1} a_f^l \right) + b_t^l
    = \sum\limits_f \left( w_{ft}^{l+1} f_h( z_f^l ) \right) + b_t^l
\end{equation}
This leads to
\begin{equation}
    \pd{z_t^{l+1}}{z_f^l}
    = \pd{}{z_f^l} \left( \sum\limits_f \left( w_{ft}^{l+1} f_h( z_f^l ) \right) + b_t^l \right)
    = w_{ft}^{l+1} \pd{f_h( z_f^l )}{z_f^l}
    = w_{ft}^{l+1} f_h^\prime( z_f^l )
    \label{eq:dzlp1_dzl}
\end{equation}
because there only remains the one term of the sum related to the specific index we derive
for. \\

Inserting equation (\ref{eq:dzlp1_dzl}) into equation (\ref{eq:delta_hidden}) we get a
formula for the error term in a hidden layer $l$
\begin{equation}
    \delta_f^l = \sum\limits_t \delta_t^{l+1} w_{ft}^{l+1} f_h^\prime( z_f^l )
    = f_h^\prime( z_f^l ) \sum\limits_t \delta_t^{l+1} w_{ft}^{l+1}
    \label{eq:delta_hidden_dep}
\end{equation}
which will be used to update the error term in the hidden layers during the backward pass.
This is called backpropagation, because the error contributions are spread layer by layer
thoughout the neural network beginning from the output layer $l=L-1$ down to the layer
$l=1$. \\

Looking at eqations (\ref{eq:loss_gradient_by_weight_condensed}) and
(\ref{eq:delta_hidden_dep}) we can see how the network layers $l-1$, $l$ and $l+1$ are
linked to each other.

\newpage